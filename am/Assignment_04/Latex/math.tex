\documentclass[]{book}

%These tell TeX which packages to use.
\usepackage{array,epsfig}
\usepackage{amsmath}
\usepackage{amsfonts}
\usepackage{amssymb}
\usepackage{amsxtra}
\usepackage{amsthm}
\usepackage{mathrsfs}
\usepackage{color}

%Here I define some theorem styles and shortcut commands for symbols I use often
\theoremstyle{definition}
\newtheorem{defn}{Definition}
\newtheorem{thm}{Theorem}
\newtheorem{cor}{Corollary}
\newtheorem*{rmk}{Remark}
\newtheorem{lem}{Lemma}
\newtheorem*{joke}{Joke}
\newtheorem{ex}{Example}
\newtheorem*{soln}{Solution}
\newtheorem{prop}{Proposition}

\newcommand{\lra}{\longrightarrow}
\newcommand{\ra}{\rightarrow}
\newcommand{\surj}{\twoheadrightarrow}
\newcommand{\graph}{\mathrm{graph}}
\newcommand{\bb}[1]{\mathbb{#1}}
\newcommand{\Z}{\bb{Z}}
\newcommand{\Q}{\bb{Q}}
\newcommand{\R}{\bb{R}}
\newcommand{\C}{\bb{C}}
\newcommand{\N}{\bb{N}}
\newcommand{\M}{\mathbf{M}}
\newcommand{\m}{\mathbf{m}}
\newcommand{\MM}{\mathscr{M}}
\newcommand{\HH}{\mathscr{H}}
\newcommand{\Om}{\Omega}
\newcommand{\Ho}{\in\HH(\Om)}
\newcommand{\bd}{\partial}
\newcommand{\del}{\partial}
\newcommand{\bardel}{\overline\partial}
\newcommand{\textdf}[1]{\textbf{\textsf{#1}}\index{#1}}
\newcommand{\img}{\mathrm{img}}
\newcommand{\ip}[2]{\left\langle{#1}
{#2}\right\rangle}
\newcommand{\inter}[1]{\mathrm{int}{#1}}
\newcommand{\exter}[1]{\mathrm{ext}{#1}}
\newcommand{\cl}[1]{\mathrm{cl}{#1}}
\newcommand{\ds}{\displaystyle}
\newcommand{\vol}{\mathrm{vol}}
\newcommand{\cnt}{\mathrm{ct}}
\newcommand{\osc}{\mathrm{osc}}
\newcommand{\LL}{\mathbf{L}}
\newcommand{\UU}{\mathbf{U}}
\newcommand{\support}{\mathrm{support}}
\newcommand{\AND}{\;\wedge\;}
\newcommand{\OR}{\;\vee\;}
\newcommand{\Oset}{\varnothing}
\newcommand{\st}{\ni}
\newcommand{\wh}{\widehat}

%Pagination stuff.
\setlength{\topmargin}{-.3 in}
\setlength{\oddsidemargin}{0in}
\setlength{\evensidemargin}{0in}
\setlength{\textheight}{9.in}
\setlength{\textwidth}{6.5in}
\pagestyle{empty}



\begin{document}


\begin{center}
{\Large Angewandte Mathematik für die Informatik \hspace{0.5cm} Sheet 4}
\textbf{Maximilian von Sternberg} %You should put your name here
\end{center}

\vspace{0.2 cm}

\begin{enumerate}
    \item \begin{enumerate}
        \item $F(x) = \frac{3}{2}\sqrt[3]{x^2} + -\frac{1}{4}\cdot sin(7 - 4x) + c$
        \item \begin{align*}
            F(x)& = \int \frac{x^2 + 1}{x^3 + 3x} \,dx \\
            & = \int \frac{x^2 + 1}{z \cdot z'} \,dz \\
            & = \int \frac{x^2 + 1}{z \cdot 3(x^2 + 1)} \,dz \\
            & = \int \frac{1}{z \cdot 3} \,dz \\
            & = \frac{1}{3} \int \frac{1}{z} \,dz \\
            & = \frac{1}{3} ln(|z|) + c \\
            & = \frac{ln(|x^3 + 3x|)}{3} + c
        \end{align*}
        \item \begin{align*}
            F(x)& = \int \frac{x - 1}{x^2 - 1} \,dx \\
            & = \int \frac{x-1}{(x+1)(x-1)} \,dx \\
            & = \int \frac{1}{x+1} \,dx \\
            & = ln(|x + 1|) + c
        \end{align*}
        \item \begin{align*}
            F(x)& = \int sin(x) \cdot x \,dx \\
            & = -x \cdot cos(x) - \int -cos(x) \, dx \\
            & = -x \cdot cos(x) + sin(x) + c
        \end{align*}
    \end{enumerate}
    \item \begin{align*}
        F(x)& = \int_0^8 \sqrt{1 + x^2} \,dx \\
        & = \int_{asinh(0))}^{asinh(8)} \sqrt{1 + (sinh(u))^2} cosh(u) \, du \\
        & = \int_{asinh(0))}^{asinh(8)} \sqrt{cosh^2(u)} cosh(u) \, du \\
        & = \int_{asinh(0))}^{asinh(8)} cosh^2(u) \, du \\
        & = [\frac{1}{2}cosh(u)sin(u) + \frac{1}{2}u]_{sinh(0)}^{sinh(8)} \\
        & = \frac{8}{2}cosh(asinh(8)) + \frac{asinh(8)}{2} \\
        & = 33,637267134
    \end{align*}
    \item \begin{align*}
         x + 2 & = x^3 - 2x + 2 \qquad |-(x + 2) \\
         0 & = x^3 - 3x \\
         0 & = x(x^2 - 3)\\
         & x_1 = 0\\
         0 & = x^2 - 3\\
        & x_{2/3} = \pm\sqrt{3}
    \end{align*}
    \begin{align*}
        A & = |\int_{-\sqrt{3}}^0 x^3 - 2x +2 - (x + 2)\, dx| + |\int_0^{-\sqrt{3}} x^3 - 2x +2 + (x + 2)\, dx| \\ 
        & = |\int_{-\sqrt{3}}^0 x^3 - 3x \, dx| + |\int_0^{-\sqrt{3}} x^3 - 3x\, dx| \\
        & = |[\frac{1}{4}x^4 - \frac{3}{2}x^2]_{-\sqrt{3}}^0| + |[\frac{1}{4}x^4 - \frac{3}{2}x^2]_0^{-\sqrt{3}}| \\
        & = \frac{9}{2}
    \end{align*}
    \item \begin{enumerate}
        \item We have to find a c which corresponds to the appropriate value that the function stays continuos. As for my hand I have provided the functions in julia. x(300) = 20550 
        \item In beschleunigung.jl
        \item The steps are only an arbitrary sample in the function. Therefore if we use a lower step size the accuracy of our guess will become worse. 
    \end{enumerate}
    \item \begin{enumerate}
        \item \begin{align*}
            V & = \int_0^8 \int_{-4}^4 -\sqrt{\frac{5}{4}} \cdot x^2 + 2y + 80\,dy\,dx \\
            & = \int_0^8 [\frac{-\sqrt{5} \cdot x^2y}{2} + y^2 + 80y]_{-4}^4\,dx \\
            & = \int_0^8 640-4\sqrt{5}\cdot x^2 \, dx \\
            & = [-\frac{4}{3}\sqrt{5} \cdot x^3 + 640x]_0^8 \\
            & = 5120-\frac{2048\sqrt{5}}{3} 
            \end{align*}
        \item \begin{align*}
            S & = \int_0^8 \int_{-4}^4 |(1, 0, \partial_xf)^T \times (0, 1, \partial_y)^T|\,dy\,dx
        \end{align*}
    \end{enumerate}
\end{enumerate}

\end{document}