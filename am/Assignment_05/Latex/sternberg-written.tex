\documentclass[]{book}

%These tell TeX which packages to use.
\usepackage{array,epsfig}
\usepackage{amsmath}
\usepackage{amsfonts}
\usepackage{amssymb}
\usepackage{amsxtra}
\usepackage{amsthm}
\usepackage{mathrsfs}
\usepackage{color}

%Here I define some theorem styles and shortcut commands for symbols I use often
\theoremstyle{definition}
\newtheorem{defn}{Definition}
\newtheorem{thm}{Theorem}
\newtheorem{cor}{Corollary}
\newtheorem*{rmk}{Remark}
\newtheorem{lem}{Lemma}
\newtheorem*{joke}{Joke}
\newtheorem{ex}{Example}
\newtheorem*{soln}{Solution}
\newtheorem{prop}{Proposition}

\newcommand{\lra}{\longrightarrow}
\newcommand{\ra}{\rightarrow}
\newcommand{\surj}{\twoheadrightarrow}
\newcommand{\graph}{\mathrm{graph}}
\newcommand{\bb}[1]{\mathbb{#1}}
\newcommand{\Z}{\bb{Z}}
\newcommand{\Q}{\bb{Q}}
\newcommand{\R}{\bb{R}}
\newcommand{\C}{\bb{C}}
\newcommand{\N}{\bb{N}}
\newcommand{\M}{\mathbf{M}}
\newcommand{\m}{\mathbf{m}}
\newcommand{\MM}{\mathscr{M}}
\newcommand{\HH}{\mathscr{H}}
\newcommand{\Om}{\Omega}
\newcommand{\Ho}{\in\HH(\Om)}
\newcommand{\bd}{\partial}
\newcommand{\del}{\partial}
\newcommand{\bardel}{\overline\partial}
\newcommand{\textdf}[1]{\textbf{\textsf{#1}}\index{#1}}
\newcommand{\img}{\mathrm{img}}
\newcommand{\ip}[2]{\left\langle{#1}
{#2}\right\rangle}
\newcommand{\inter}[1]{\mathrm{int}{#1}}
\newcommand{\exter}[1]{\mathrm{ext}{#1}}
\newcommand{\cl}[1]{\mathrm{cl}{#1}}
\newcommand{\ds}{\displaystyle}
\newcommand{\vol}{\mathrm{vol}}
\newcommand{\cnt}{\mathrm{ct}}
\newcommand{\osc}{\mathrm{osc}}
\newcommand{\LL}{\mathbf{L}}
\newcommand{\UU}{\mathbf{U}}
\newcommand{\support}{\mathrm{support}}
\newcommand{\AND}{\;\wedge\;}
\newcommand{\OR}{\;\vee\;}
\newcommand{\Oset}{\varnothing}
\newcommand{\st}{\ni}
\newcommand{\wh}{\widehat}

%Pagination stuff.
\setlength{\topmargin}{-.3 in}
\setlength{\oddsidemargin}{0in}
\setlength{\evensidemargin}{0in}
\setlength{\textheight}{9.in}
\setlength{\textwidth}{6.5in}
\pagestyle{empty}



\begin{document}


\begin{center}
{\Large Angewandte Mathematik in der Informatik \hspace{0.5cm} Sheet 5}
\textbf{Maximilian von Sternberg} %You should put your name here
\end{center}

\vspace{0.2 cm}

\begin{enumerate}
    \item \begin{enumerate}
        \item Yes, it is a valid solution:
        \begin{align*}
            & y^{\prime \prime} + 2y^{\prime} = 3y \quad |- 2y^{\prime} \\
            \Leftrightarrow & \; y^{\prime \prime} = 3y - 2y^{\prime} \\
            = & \; (e^{-3x})^{\prime \prime} = 3 \cdot e^{-3x} - 2(e^{-3x})^{\prime} \\
            = & \; 9 \cdot e^{-3x} = 3 \cdot e^{-3x} - 2(e^{-3x})^{\prime} \\
            = & \; 9 \cdot e^{-3x} = 3 \cdot e^{-3x} + 6 \cdot e^{-3x} \\
            = & \; 9 \cdot e^{-3x} = 9 \cdot e^{-3x} 
        \end{align*}
        \item \begin{align*}
            & x^2y^{\prime \prime} - xy^{\prime} + 2y = 0 \quad |- x^2y^{\prime \prime} ; \cdot -1 \\
            \Leftrightarrow & \; x^2y^{\prime \prime} = xy^{\prime} - 2y \\
            = & \; x^2(x \cos(ln |x|))^{\prime \prime} = x(x \cos(ln |x|))^{\prime} - 2(x \cos(ln |x|)) \\
            = & \; -\dfrac{x^2 \cdot \sin(\ln(|x|)+\cos(\ln(|x|))}{x}) = x(x \cos(ln |x|))^{\prime} - 2(x \cos(ln |x|)) \\
            = & \; x \sin(\ln|x|) + x \cos(\ln|x|) = x(x \cos(ln |x|))^{\prime} - 2(x \cos(ln |x|)) \\
            = & \; - x \sin(\ln|x|) - x \cos(\ln|x|) = - x \sin(ln |x|) + x \cos(ln |x| - 2(x \cos(ln |x|)) \\
            = & \; - x \sin(\ln|x|) - x \cos(\ln|x|) = - x \sin(ln |x|) - x \cos(ln |x|) \\
        \end{align*}
        \item \begin{align*}
            & xy^{\prime} - 3y = x^3 \quad | + 3y ; \frac{1}{x}\\
            \Leftrightarrow & \; (x^3(C + ln |x|))^{\prime} = x^2 + \frac{1}{x} 3(x^3(C + ln |x|)) \\
            = & \; x^2 + 3x^2(C + ln |x|) = x^2 + \frac{1}{x} 3(x^3(C + ln |x|)) \\
            = & \; x^2 + 3x^2(C + ln |x|) = x^2 + \frac{3x^3(C + ln |x|)}{x}  \\
            = & \; x^2 + 3x^2(C + ln |x|) = x^2 + 3x^2(C + ln |x|)  \\
        \end{align*}
        \begin{align*}
            17 = & \; 1^3 \cdot (C + ln 1) = C\\
            \Rightarrow & \; x^3 \cdot (17 + ln |x|) 
        \end{align*}
    \end{enumerate}
    \item \begin{align*}
        y^{\prime} = & \; \frac{x + e^{2x}}{y} \\
        \frac{dy}{dx} = & \; \frac{x + e^{2x}}{y} \quad |\cdot y ; \cdot dx \\
        y \cdot dy = & \; x + e^{2x} \cdot dx \quad |\int \\
        \frac{1}{2} y^2 = & \; \frac{1}{2}x^2 + \frac{1}{2} \cdot e^{2x} \quad |\cdot 2 ; \sqrt{} \\
        y = & \; \sqrt{x^2 + e^{2x}}
    \end{align*}
    \item \begin{enumerate}
        \item \begin{align*}
            \dot{y}_0 & = 2\sqrt{2} \cos(\frac{\pi}{4}) = 2\\
            \dot{x}_0 & = 2\sqrt{2} \sin(\frac{\pi}{4}) = 2\\
            y_0 & = 1 \\
            x_0 & = 0 \\
            \\
            \dot{y}_1 & =  -g \cdot h + \dot{y}_0 = -\frac{5}{2} + 2 = -0,5 \\
            \dot{x}_1 & =  \dot{x}_0 = 2 \\
            \dot{y}_2 & =  -\frac{g}{4} + \dot{y}_1 = -\frac{5}{2} -0,5 = -3 \\
            \dot{x}_2 & =  \dot{x}_1 = 2 \\
            \\
            y_1 & = h \cdot \dot{y_0} + y_0 = \frac{1}{2} + 1 = \frac{3}{2}\\
            x_1 & = h \cdot \dot{x_0} + x_0 = \frac{1}{2} \\
            y_2 & = h \cdot \dot{y_1} + y_1 = -\frac{1}{8} + \frac{3}{2} = \frac{11}{8}\\
            x_2 & = h \cdot \dot{x_1} + x_1 = 1 \\
            \\
        \end{align*}
        \item Julia
        \item Julia
        \item The bigger h gets, the steeper the error curve gets
    \end{enumerate}
\end{enumerate}

\end{document}