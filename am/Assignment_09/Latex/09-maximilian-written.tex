\documentclass[]{book}

%These tell TeX which packages to use.
\usepackage{array,epsfig}
\usepackage{amsmath}
\usepackage{amsfonts}
\usepackage{amssymb}
\usepackage{amsxtra}
\usepackage{amsthm}
\usepackage{mathrsfs}
\usepackage{color}

%Here I define some theorem styles and shortcut commands for symbols I use often
\theoremstyle{definition}
\newtheorem{defn}{Definition}
\newtheorem{thm}{Theorem}
\newtheorem{cor}{Corollary}
\newtheorem*{rmk}{Remark}
\newtheorem{lem}{Lemma}
\newtheorem*{joke}{Joke}
\newtheorem{ex}{Example}
\newtheorem*{soln}{Solution}
\newtheorem{prop}{Proposition}

\newcommand{\lra}{\longrightarrow}
\newcommand{\ra}{\rightarrow}
\newcommand{\surj}{\twoheadrightarrow}
\newcommand{\graph}{\mathrm{graph}}
\newcommand{\bb}[1]{\mathbb{#1}}
\newcommand{\Z}{\bb{Z}}
\newcommand{\Q}{\bb{Q}}
\newcommand{\R}{\bb{R}}
\newcommand{\C}{\bb{C}}
\newcommand{\N}{\bb{N}}
\newcommand{\M}{\mathbf{M}}
\newcommand{\m}{\mathbf{m}}
\newcommand{\MM}{\mathscr{M}}
\newcommand{\HH}{\mathscr{H}}
\newcommand{\Om}{\Omega}
\newcommand{\Ho}{\in\HH(\Om)}
\newcommand{\bd}{\partial}
\newcommand{\del}{\partial}
\newcommand{\bardel}{\overline\partial}
\newcommand{\textdf}[1]{\textbf{\textsf{#1}}\index{#1}}
\newcommand{\img}{\mathrm{img}}
\newcommand{\ip}[2]{\left\langle{#1}
{#2}\right\rangle}
\newcommand{\inter}[1]{\mathrm{int}{#1}}
\newcommand{\exter}[1]{\mathrm{ext}{#1}}
\newcommand{\cl}[1]{\mathrm{cl}{#1}}
\newcommand{\ds}{\displaystyle}
\newcommand{\vol}{\mathrm{vol}}
\newcommand{\cnt}{\mathrm{ct}}
\newcommand{\osc}{\mathrm{osc}}
\newcommand{\LL}{\mathbf{L}}
\newcommand{\UU}{\mathbf{U}}
\newcommand{\support}{\mathrm{support}}
\newcommand{\AND}{\;\wedge\;}
\newcommand{\OR}{\;\vee\;}
\newcommand{\Oset}{\varnothing}
\newcommand{\st}{\ni}
\newcommand{\wh}{\widehat}

%Pagination stuff.
\setlength{\topmargin}{-.3 in}
\setlength{\oddsidemargin}{0in}
\setlength{\evensidemargin}{0in}
\setlength{\textheight}{9.in}
\setlength{\textwidth}{6.5in}
\pagestyle{empty}



\begin{document}


\begin{center}
{\Large Ami \hspace{0.5cm} Sheet 9}
\textbf{Maximilian von Sternberg} %You should put your name here
\end{center}

\vspace{0.2 cm}

\begin{enumerate}
    \item \begin{enumerate}
        \item \begin{align*}
            &
            \begin{pmatrix}
                1 & 0 & 0 & 0 & 0 \\
                0 & 1 & 0 & 0 & 0 \\
                0 & 0 & 1 & 0 & 0 \\
                0 & 0 & 0 & 1 & 0 \\
                0 & 0 & 0 & 0 & 1 
            \end{pmatrix}
            \cdot
            \begin{pmatrix}
                1 & 3 & 1 & 2 & 5 \\
                3 & 13 & 7 & 8 & 17 \\
                1 & 7 & 21 & 8 & 15 \\
                2 & 8 & 8 & 7 & 16 \\
                5 & 17 & 15 & 16 & 40 
            \end{pmatrix}
            =
            \begin{pmatrix}
                1 & 0 & 0 & 0 & 0 \\
                3 & 1 & 0 & 0 & 0 \\
                1 & 0 & 1 & 0 & 0 \\
                2 & 0 & 0 & 1 & 0 \\
                5 & 0 & 0 & 0 & 1 
            \end{pmatrix} 
            \cdot
            \begin{pmatrix}
                1 & 3 & 1 & 2 & 5 \\
                0 & 4 & 4 & 2 & 2 \\
                0 & 4 & 20 & 6 & 10 \\
                0 & 2 & 6 & 3 & 6 \\
                0 & 2 & 10 & 6 & 15 
            \end{pmatrix} \\
            =
            &
            \begin{pmatrix}
                1 & 0 & 0 & 0 & 0 \\
                3 & 1 & 0 & 0 & 0 \\
                1 & 1 & 1 & 0 & 0 \\
                2 & \frac{1}{2} & 0 & 1 & 0 \\
                5 & \frac{1}{2} & 0 & 0 & 1 
            \end{pmatrix} 
            \cdot
            \begin{pmatrix}
                1 & 3 & 1 & 2 & 5 \\
                0 & 4 & 4 & 2 & 2 \\
                0 & 0 & 16 & 4 & 8 \\
                0 & 0 & 4 & 2 & 5 \\
                0 & 0 & 8 & 5 & 14 
            \end{pmatrix} 
            =
            \begin{pmatrix}
                1 & 0 & 0 & 0 & 0 \\
                3 & 1 & 0 & 0 & 0 \\
                1 & 1 & 1 & 0 & 0 \\
                2 & \frac{1}{2} & \frac{1}{4} & 1 & 0 \\
                5 & \frac{1}{2} & \frac{1}{2} & 0 & 1 
            \end{pmatrix} 
            \cdot
            \begin{pmatrix}
                1 & 3 & 1 & 2 & 5 \\
                0 & 4 & 4 & 2 & 2 \\
                0 & 0 & 16 & 4 & 8 \\
                0 & 0 & 0 & 1 & 3 \\
                0 & 0 & 0 & 3 & 10 
            \end{pmatrix} \\
            =
            &
            \begin{pmatrix}
                1 & 0 & 0 & 0 & 0 \\
                3 & 1 & 0 & 0 & 0 \\
                1 & 1 & 1 & 0 & 0 \\
                2 & \frac{1}{2} & \frac{1}{4} & 1 & 0 \\
                5 & \frac{1}{2} & \frac{1}{2} & 3 & 1 
            \end{pmatrix} 
            \cdot
            \begin{pmatrix}
                1 & 3 & 1 & 2 & 5 \\
                0 & 4 & 4 & 2 & 2 \\
                0 & 0 & 16 & 4 & 8 \\
                0 & 0 & 0 & 1 & 3 \\
                0 & 0 & 0 & 0 & 1 
            \end{pmatrix}
        \end{align*}
        \item The matrix is symmetric and regular
        \item $$\begin{pmatrix}
            1 & 0 & 0 & 0 & 0 \\
            3 & 2 & 0 & 0 & 0 \\
            1 & 2 & 4 & 0 & 0 \\
            2 & 1 & 1 & 1 & 0 \\
            5 & 1 & 2 & 3 & 1
        \end{pmatrix}$$
        \item
            $$\begin{pmatrix}
                1 & 0 & 0 & 0 & 0 \\
                3 & 2 & 0 & 0 & 0 \\
                1 & 2 & 4 & 0 & 0 \\
                2 & 1 & 1 & 1 & 0 \\
                5 & 1 & 2 & 3 & 1
            \end{pmatrix}
            \cdot 
            y
            =
            (1 1 1 1 1)^T$$
        $$y = \begin{pmatrix} 1 \\ -1 \\ \frac{1}{2} \\ -\frac{1}{2} \\ -\frac{5}{2} \end{pmatrix}$$
        $$ \begin{pmatrix}
            1 & 3 & 1 & 2 & 5 \\
            0 & 2 & 2 & 1 & 1 \\
            0 & 0 & 4 & 1 & 2 \\
            0 & 0 & 0 & 1 & 3 \\
            0 & 0 & 0 & 0 & 1
        \end{pmatrix}
        \cdot 
        x
        =
        \begin{pmatrix} 1 \\ -1 \\ \frac{1}{2} \\ -\frac{1}{2} \\ -\frac{5}{2} \end{pmatrix}$$
        $$x = \begin{pmatrix} 7 \\ -\frac{19}{8} \\ -\frac{3}{8} \\ 7 \\ -\frac{5}{2} \end{pmatrix}$$
        \item If you direktly wanted to calculate, you would have to use the Gauss algorithm, to transform the matrix into a unit matrix and perform these transformations on another unit matrix. The unit matrix will then transform into the inverse of the matrix.
    \end{enumerate}
    \item \begin{enumerate}
        \item $$x_1 = \frac{1}{1}(1 - (3 + 1 + 2 + 5)) = -10$$
        $$x_2 = \frac{1}{13}(1 - (3 + 7 + 8 + 17)) = 2.26$$
        $$x_3 = \frac{1}{21}(1 - (1 + 7 + 8 + 15)) = 1.43$$
        $$x_4 = \frac{1}{7}(1 - (2 + 8 + 8 + 16)) = 3.29$$
        $$x_5 = \frac{1}{40}(1 - (5 + 17 + 15 + 16)) = 1.3$$
        \item Julia
        \item Julia
        \item 
    \end{enumerate}
    \item \begin{enumerate}
        \item 
    \end{enumerate}
\end{enumerate}

\end{document}