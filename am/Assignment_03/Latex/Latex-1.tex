% --------------------------------------------------------------
% This is all preamble stuff that you don't have to worry about.
% Head down to where it says "Start here"
% --------------------------------------------------------------
 
\documentclass[12pt]{article}
 
\usepackage[margin=1in]{geometry} 
\usepackage{amsmath,amsthm,amssymb}
 
\newcommand{\N}{\mathbb{N}}
\newcommand{\Z}{\mathbb{Z}}
 
\newenvironment{theorem}[2][Theorem]{\begin{trivlist}
\item[\hskip \labelsep {\bfseries #1}\hskip \labelsep {\bfseries #2.}]}{\end{trivlist}}
\newenvironment{lemma}[2][Lemma]{\begin{trivlist}
\item[\hskip \labelsep {\bfseries #1}\hskip \labelsep {\bfseries #2.}]}{\end{trivlist}}
\newenvironment{exercise}[2][Exercise]{\begin{trivlist}
\item[\hskip \labelsep {\bfseries #1}\hskip \labelsep {\bfseries #2.}]}{\end{trivlist}}
\newenvironment{problem}[2][Problem]{\begin{trivlist}
\item[\hskip \labelsep {\bfseries #1}\hskip \labelsep {\bfseries #2.}]}{\end{trivlist}}
\newenvironment{question}[2][Question]{\begin{trivlist}
\item[\hskip \labelsep {\bfseries #1}\hskip \labelsep {\bfseries #2.}]}{\end{trivlist}}
\newenvironment{corollary}[2][Corollary]{\begin{trivlist}
\item[\hskip \labelsep {\bfseries #1}\hskip \labelsep {\bfseries #2.}]}{\end{trivlist}}
 
\begin{document}
 
% --------------------------------------------------------------
%                         Start here
% --------------------------------------------------------------
 
\title{Assignment 3}%replace X with the appropriate number
\author{Maximilian von Sternberg\\ %replace with your name
Angewandte Mathematik} %if necessary, replace with your course title
 
\maketitle
 
\begin{exercise}
1 
\begin{question}
a
%Note 1: The * tells LaTeX not to number the lines.  If you remove the *, be sure to remove it below, too.
%Note 2: Inside the align environment, you do not want to use $-signs.  The reason for this is that this is already a math environment. This is why we have to include \text{} around any text inside the align environment.
\begin{align*}
f(x,y )& = -4x^2 + 2 * (2x - y)^3 - (x + y)^3 + x + 4y^2 - y - 4\\ 
 & = 2 * (2x - y)^3 - (x + y)^3 + x  + 4y^2 - 4x^2 - y - 4 \\	
 & = 2 * (8x^3 - y^3 -12x^2y + 6xy^2) - (x^3 + 3x^2y + 3xy^2 + y^3) + x  + 4y^2 - 4x^2 - y - 4 \\	
 & = 16x^3 - 2y^3 - 24x^2y + 12xy^2 - x^3 - 3x^2y - 3xy^2 - y^3 + x  + 4y^2 - 4x^2 - y - 4 \\	
 & = 15x^3 - 3y^3 - 27x^2y + 9xy^2 - 4x^2  + 4y^2 + x  - y - 4 
\end{align*}

\begin{align*}
\frac{\partial f(x, y)}{\partial x}& = 45x^2 - 54yx + 9y^2 - 8x  + 1 
\end{align*}

\begin{align*}
\frac{\partial f(x, y)}{\partial x}& = -9y^2 - 27x^2 + 18xy + 8y - 1
\end{align*}

\end{question}

\begin{question}
b

\begin{align*}
0 = \frac{\partial f(0, y)}{\partial y}& = -9y^2 + 8y - 1\\
\text{Quadratic-Formula}: y& =  \frac{-8\pm\sqrt{8^2-4(-9)(-1)}}{2(-9)}\\
& = \frac{-8\pm\sqrt{64-36}}{-18}\\
y_1 = 0,738 \\
y_2 = 0,15
\end{align*}

\begin{align*}
0 = \frac{\partial f(x, -1/2)}{\partial x}& = 45x^2 + 35x + 3,25\\
\text{Quadratic-Formula}: y& =  \frac{-35\pm\sqrt{35^2 - 4 * 43 * 3,25}}{2*45}\\
& = \frac{-35\pm\sqrt{1225-559}}{90}\\
x_1 = -0,108 \\
x_2 = -0,67
\end{align*}

\end{question}
\begin{question}
c
\begin{align*}
\triangledown f (-\frac{1}{2}, -\frac{1}{2}) & = (5, -9,5) \\
\triangledown f (\frac{1}{2}, \frac{1}{2}) & = (-3, -1,5) \\
\triangledown f (-\frac{1}{2}, \frac{1}{2}) & = (32 , -10,5) \\
\triangledown f (\frac{1}{2}, -\frac{1}{2}) & = (24, -18,5) \\
\triangledown f (-\frac{1}{2}, -\frac{1}{2}) & = (3,25, -7,25)
\end{align*}
\end{question}
\end{exercise}
 
\begin{exercise}
2
The distance is $x(t) = v_x * t$. Now we need to maximize the time for which $y(t) > 0$ and where $v_x$ is sufficiently big.  We can combine these two properties by making $v_x$ and $v_y$ dependent on $\theta$.
\begin{align}
\begin{pmatrix}
 v * cos \theta\\
v * sin \theta
\end{pmatrix}
\end{align}
the best $t$ is now the non trivial solution for $0 = tv * sin \theta - \frac{gt^2}{2}$

\begin{align*}
0 & = tv_y * sin \theta - \frac{gt^2}{2} \quad |+\frac{gt^2}{2}\\
& \Leftrightarrow \frac{gt^2}{2} = tv_y * sin \theta \quad |* \frac{1}{vsin \theta} \\
& \Leftrightarrow \frac{gt^2}{2v * sin \theta} = t 
\end{align*}

Now that we have the zero dependant on $t$ we can combine the three functions to the length function 
$$f(\theta) = v*cos \theta * \frac{gt^2}{2v * sin \theta}$$ \\

To get the min of this function we just have to finde $0 = f'(\theta)$

\end{exercise}

\begin{exercise}
3
\begin{question}
b
Because f'(x) = 0 you would divide by zero which is impossible
\end{question}

\begin{question}
c
If you use f' for the function the roots will be the local minima of f
\end{question}


\end{exercise}

% --------------------------------------------------------------
%     You don't have to mess with anything below this line.
% --------------------------------------------------------------
 
\end{document}