\documentclass[]{book}

%These tell TeX which packages to use.
\usepackage{array,epsfig}
\usepackage{amsmath}
\usepackage{amsfonts}
\usepackage{amssymb}
\usepackage{amsxtra}
\usepackage{amsthm}
\usepackage{mathrsfs}
\usepackage{color}

%Here I define some theorem styles and shortcut commands for symbols I use often
\theoremstyle{definition}
\newtheorem{defn}{Definition}
\newtheorem{thm}{Theorem}
\newtheorem{cor}{Corollary}
\newtheorem*{rmk}{Remark}
\newtheorem{lem}{Lemma}
\newtheorem*{joke}{Joke}
\newtheorem{ex}{Example}
\newtheorem*{soln}{Solution}
\newtheorem{prop}{Proposition}

\newcommand{\lra}{\longrightarrow}
\newcommand{\ra}{\rightarrow}
\newcommand{\surj}{\twoheadrightarrow}
\newcommand{\graph}{\mathrm{graph}}
\newcommand{\bb}[1]{\mathbb{#1}}
\newcommand{\Z}{\bb{Z}}
\newcommand{\Q}{\bb{Q}}
\newcommand{\R}{\bb{R}}
\newcommand{\C}{\bb{C}}
\newcommand{\N}{\bb{N}}
\newcommand{\M}{\mathbf{M}}
\newcommand{\m}{\mathbf{m}}
\newcommand{\MM}{\mathscr{M}}
\newcommand{\HH}{\mathscr{H}}
\newcommand{\Om}{\Omega}
\newcommand{\Ho}{\in\HH(\Om)}
\newcommand{\bd}{\partial}
\newcommand{\del}{\partial}
\newcommand{\bardel}{\overline\partial}
\newcommand{\textdf}[1]{\textbf{\textsf{#1}}\index{#1}}
\newcommand{\img}{\mathrm{img}}
\newcommand{\ip}[2]{\left\langle{#1}
{#2}\right\rangle}
\newcommand{\inter}[1]{\mathrm{int}{#1}}
\newcommand{\exter}[1]{\mathrm{ext}{#1}}
\newcommand{\cl}[1]{\mathrm{cl}{#1}}
\newcommand{\ds}{\displaystyle}
\newcommand{\vol}{\mathrm{vol}}
\newcommand{\cnt}{\mathrm{ct}}
\newcommand{\osc}{\mathrm{osc}}
\newcommand{\LL}{\mathbf{L}}
\newcommand{\UU}{\mathbf{U}}
\newcommand{\support}{\mathrm{support}}
\newcommand{\AND}{\;\wedge\;}
\newcommand{\OR}{\;\vee\;}
\newcommand{\Oset}{\varnothing}
\newcommand{\st}{\ni}
\newcommand{\wh}{\widehat}

%Pagination stuff.
\setlength{\topmargin}{-.3 in}
\setlength{\oddsidemargin}{0in}
\setlength{\evensidemargin}{0in}
\setlength{\textheight}{9.in}
\setlength{\textwidth}{6.5in}
\pagestyle{empty}



\begin{document}


\begin{center}
{\Large Angwandte Mathematik in der Informatik \hspace{0.5cm} Sheet 10}
\textbf{Maximilian von Sternberg} %You should put your name here
\end{center}

\vspace{0.2 cm}

\begin{enumerate}
    \item Underfitting is, when you dont represent a dataset adequatly. It is only a loose representation of the data and the error is really big. For us you can see this in the linear fit. \\ Overfitting can be when the function is to specific to the dataset this would make it impossible to make future prediction. 
    \item 2
    \item \begin{enumerate}
        \item \begin{align*}
            \begin{pmatrix}
                -1 & 1 & -1 & 1 \\
                3 & -2 & 1 & 0 \\
                1 & 1 & 1 & 1 \\
                3 & 2 & 1 & 0 
            \end{pmatrix}
            \cdot x = \begin{pmatrix}
                0 \\ 
                -5 \\
                10 \\
                20
            \end{pmatrix}\\
            \Leftrightarrow x = \begin{pmatrix}
                \frac{5}{4} \\
                \frac{25}{4} \\
                \frac{15}{4} \\
                \frac{-5}{4} \\
            \end{pmatrix}
            \\
            \Rightarrow \frac{5}{4}x^3 + \frac{25}{4}x^2 +\frac{15}{4}x +\frac{-5}{4}
        \end{align*}
        \item \begin{align*}
            l_0 = \frac{(x - 2.75)(x -4)}{(-0.75)(-2)} = \frac{11 - 6.75 x + x^2}{1.5}            l_0 = \frac{(x - 2.75)(x -4)}{(-0.75)(-2)} = \frac{11 - 6.75 x + x^2}{1.5}
        \end{align*}
        \item \begin{align*}
            & \text{Interpolation in x: } g(x, y) = f(0, y) + (1 - x) \cdot f(1, y) \\
            \Rightarrow & \text{Interpolation in y: } f(x, y) = g(x, 0) + (1 - y) \cdot g(x, 1) \\ 
            = & f(0, 0) + (1 - x) \cdot f(1, 0) + (1 - y) \cdot (f(0, 1) + (1 - x) \cdot f(1, 1)) \\
            = & f(x, y) = \begin{pmatrix}
                1 - x \\
                x
            \end{pmatrix}^T
            \begin{pmatrix}
                f(0, 0) && f(0, 1)\\
                f(1, 0) && f(1, 1)
            \end{pmatrix}
            \begin{pmatrix}
                1 - y \\
                y
            \end{pmatrix} \\
            & \text{Interpolation in y: } g(x, y) = f(x, 0) + (1 - y) \cdot f(x, 1) \\
            \Rightarrow & \text{Interpolation in x: } f(x, y) = g(0, y) + (1 - x) \cdot g(1, y) \\
            = & f(0, 0) + (1 - y) \cdot f(1, 0) + (1 - x) \cdot (f(0, 1) + (1 - y) \cdot f(1, 1)) \\
            = & f(x, y) = \begin{pmatrix}
                1 - x \\
                x
            \end{pmatrix}^T
            \begin{pmatrix}
                f(0, 0) && f(0, 1)\\
                f(1, 0) && f(1, 1)
            \end{pmatrix}
            \begin{pmatrix}
                1 - y \\
                y
            \end{pmatrix} \square
        \end{align*}
    \end{enumerate}
\end{enumerate}

\end{document}