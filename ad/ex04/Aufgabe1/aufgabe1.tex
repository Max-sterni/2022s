\documentclass[]{book}

%These tell TeX which packages to use.
\usepackage{array,epsfig}
\usepackage{amsmath}
\usepackage{amsfonts}
\usepackage{amssymb}
\usepackage{amsxtra}
\usepackage{amsthm}
\usepackage{mathrsfs}
\usepackage{color}

%Here I define some theorem styles and shortcut commands for symbols I use often
\theoremstyle{definition}
\newtheorem{defn}{Definition}
\newtheorem{thm}{Theorem}
\newtheorem{cor}{Corollary}
\newtheorem*{rmk}{Remark}
\newtheorem{lem}{Lemma}
\newtheorem*{joke}{Joke}
\newtheorem{ex}{Example}
\newtheorem*{soln}{Solution}
\newtheorem{prop}{Proposition}

\newcommand{\lra}{\longrightarrow}
\newcommand{\ra}{\rightarrow}
\newcommand{\surj}{\twoheadrightarrow}
\newcommand{\graph}{\mathrm{graph}}
\newcommand{\bb}[1]{\mathbb{#1}}
\newcommand{\Z}{\bb{Z}}
\newcommand{\Q}{\bb{Q}}
\newcommand{\R}{\bb{R}}
\newcommand{\C}{\bb{C}}
\newcommand{\N}{\bb{N}}
\newcommand{\M}{\mathbf{M}}
\newcommand{\m}{\mathbf{m}}
\newcommand{\MM}{\mathscr{M}}
\newcommand{\HH}{\mathscr{H}}
\newcommand{\Om}{\Omega}
\newcommand{\Ho}{\in\HH(\Om)}
\newcommand{\bd}{\partial}
\newcommand{\del}{\partial}
\newcommand{\bardel}{\overline\partial}
\newcommand{\textdf}[1]{\textbf{\textsf{#1}}\index{#1}}
\newcommand{\img}{\mathrm{img}}
\newcommand{\ip}[2]{\left\langle{#1}
{#2}\right\rangle}
\newcommand{\inter}[1]{\mathrm{int}{#1}}
\newcommand{\exter}[1]{\mathrm{ext}{#1}}
\newcommand{\cl}[1]{\mathrm{cl}{#1}}
\newcommand{\ds}{\displaystyle}
\newcommand{\vol}{\mathrm{vol}}
\newcommand{\cnt}{\mathrm{ct}}
\newcommand{\osc}{\mathrm{osc}}
\newcommand{\LL}{\mathbf{L}}
\newcommand{\UU}{\mathbf{U}}
\newcommand{\support}{\mathrm{support}}
\newcommand{\AND}{\;\wedge\;}
\newcommand{\OR}{\;\vee\;}
\newcommand{\Oset}{\varnothing}
\newcommand{\st}{\ni}
\newcommand{\wh}{\widehat}

%Pagination stuff.
\setlength{\topmargin}{-.3 in}
\setlength{\oddsidemargin}{0in}
\setlength{\evensidemargin}{0in}
\setlength{\textheight}{9.in}
\setlength{\textwidth}{6.5in}
\pagestyle{empty}



\begin{document}


\begin{center}
{\Large Algorithmen und Datenstrukturen \hspace{0.5cm} Sheet 4}
\textbf{Maximilian von Sternberg} %You should put your name here
\end{center}

\vspace{0.2 cm}

\begin{enumerate}
    \item Immer wenn das Array die hälfte seiner ursprünglichen größe erreicht hat, muss man alle Elemente auf die linke Seite verschieben und die rechte Seite freigeben. Dies hat genau diesselbe Laufzeit wie das Kopieren, da genau wie beim Kopieren im schlimmsten fall die Shift Operation O(n) dauert.
    \item \begin{enumerate}
        \item 10
        \item 10 - 1
        \item 5 - 8 - 10 - 1
        \item 5 - 2 - 8 - 10 - 1
        \item 5 - 2 - 8 - 9 - 10 - 1
        \item 2 - 8 - 9 - 10 - 1
    \end{enumerate}
    \begin{enumerate}
        \item Da das aufrufen eines Elementes einer verketteten Liste O(i) ist, wäre eine Schleife die über die Liste läuft $O(n^2)$, da sie für jedes Element i Aktionen durchführt. Nutzt man jedoch einen Itererator hat die Schleife $O(n)$, weil die next Funktion $O(1)$ ist.
        \item Man muss eine neue Liste erstellen und die daten in dieses Kopieren. Nun kann man  auf der Kopierten liste iterieren. Das Kopieren braucht dann O(n)
        \item Das die Funktion den Index des Iterators mit dem Shift wieder zurück auf das richtige Element bewegt.
    \end{enumerate} 
\end{enumerate}

\end{document}