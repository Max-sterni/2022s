\documentclass[]{book}

%These tell TeX which packages to use.
\usepackage{array,epsfig}
\usepackage{amsmath}
\usepackage{amsfonts}
\usepackage{amssymb}
\usepackage{amsxtra}
\usepackage{amsthm}
\usepackage{mathrsfs}
\usepackage{color}

%Here I define some theorem styles and shortcut commands for symbols I use often
\theoremstyle{definition}
\newtheorem{defn}{Definition}
\newtheorem{thm}{Theorem}
\newtheorem{cor}{Corollary}
\newtheorem*{rmk}{Remark}
\newtheorem{lem}{Lemma}
\newtheorem*{joke}{Joke}
\newtheorem{ex}{Example}
\newtheorem*{soln}{Solution}
\newtheorem{prop}{Proposition}

\newcommand{\lra}{\longrightarrow}
\newcommand{\ra}{\rightarrow}
\newcommand{\surj}{\twoheadrightarrow}
\newcommand{\graph}{\mathrm{graph}}
\newcommand{\bb}[1]{\mathbb{#1}}
\newcommand{\Z}{\bb{Z}}
\newcommand{\Q}{\bb{Q}}
\newcommand{\R}{\bb{R}}
\newcommand{\C}{\bb{C}}
\newcommand{\N}{\bb{N}}
\newcommand{\M}{\mathbf{M}}
\newcommand{\m}{\mathbf{m}}
\newcommand{\MM}{\mathscr{M}}
\newcommand{\HH}{\mathscr{H}}
\newcommand{\Om}{\Omega}
\newcommand{\Ho}{\in\HH(\Om)}
\newcommand{\bd}{\partial}
\newcommand{\del}{\partial}
\newcommand{\bardel}{\overline\partial}
\newcommand{\textdf}[1]{\textbf{\textsf{#1}}\index{#1}}
\newcommand{\img}{\mathrm{img}}
\newcommand{\ip}[2]{\left\langle{#1}
{#2}\right\rangle}
\newcommand{\inter}[1]{\mathrm{int}{#1}}
\newcommand{\exter}[1]{\mathrm{ext}{#1}}
\newcommand{\cl}[1]{\mathrm{cl}{#1}}
\newcommand{\ds}{\displaystyle}
\newcommand{\vol}{\mathrm{vol}}
\newcommand{\cnt}{\mathrm{ct}}
\newcommand{\osc}{\mathrm{osc}}
\newcommand{\LL}{\mathbf{L}}
\newcommand{\UU}{\mathbf{U}}
\newcommand{\support}{\mathrm{support}}
\newcommand{\AND}{\;\wedge\;}
\newcommand{\OR}{\;\vee\;}
\newcommand{\Oset}{\varnothing}
\newcommand{\st}{\ni}
\newcommand{\wh}{\widehat}

%Pagination stuff.
\setlength{\topmargin}{-.3 in}
\setlength{\oddsidemargin}{0in}
\setlength{\evensidemargin}{0in}
\setlength{\textheight}{9.in}
\setlength{\textwidth}{6.5in}
\pagestyle{empty}



\begin{document}


\begin{center}
{\Large  Angewandte Mathematik in der Informatik \hspace{0.5cm} Sheet 6}
\textbf{Maximilian von Sternberg} %You should put your name here
\end{center}

\vspace{0.2 cm}

\begin{enumerate}
    \item \begin{enumerate}
        \item \begin{align*}
            \int_0^\pi \sqrt{r^2 + (\frac{dr}{d\theta})^2} \, d\theta &= \int_0^\pi \sqrt{(1 - \cos \theta)^2 + (\sin \theta)^2} \, d\theta \\
            & = \int_0^\pi \sqrt{1 - 2\cos\theta + \cos^2\theta + \sin^2 \theta} \, d\theta \\
            & = \int_0^\pi \sqrt{- 2\cos\theta + 2} \, d\theta \\
            & = \int_0^\pi \sqrt{- 2(2\cos^2 \frac{\theta}{2} - 1) + 2} \, d\theta \\
            & = \int_0^\pi \sqrt{- 4\cos^2 \frac{\theta}{2} + 4} \, d\theta \\
            & = \int_0^\pi \sqrt{4(1 + - \cos^2 \frac{\theta}{2})} \, d\theta \\
            & = \int_0^\pi \sqrt{4\sin^2 \frac{\theta}{2}} \, d\theta \\
            & = \int_0^\pi 2\sin \frac{\theta}{2} \, d\theta \\
            & = [-4\cos \frac{\theta}{2}]_0^\pi \\
            & = -4 \cos \frac{\pi}{2} + 4\cos 0 \\
            & = 4        
        \end{align*}
            \item \begin{align*}
                \int_0^{2\pi} \sqrt{ \sum_{i=1}^{n} (\gamma^\prime_i(\theta))^2 } & = \int_0^{2\pi} \sqrt{(2\cos \theta + 2)^2 + (-2\sin\theta)^2} \, d\theta \\
                & = \int_0^{2\pi} \sqrt{4\cos^2 \theta + 8\cos \theta + 4 + 4 \sin^2\theta} \, d\theta \\
                & = \int_0^{2\pi} \sqrt{8\cos \theta + 8} \, d\theta \\
                & = \int_0^{2\pi} \sqrt{8(\cos \theta + 1)} \, d\theta \\
                & = \int_0^{2\pi} \sqrt{8 \cdot 2\cos^2 \frac{\theta}{2}} \, d\theta \\
                & = 4 \cdot \int_0^{2\pi} | \cos \frac{\theta}{2}| \, d\theta \\
                & = 4 \cdot \int_0^{\pi} |\cos u| \, \frac{du}{du^\prime} \\
                & = 8 \cdot \int_0^{\pi} |\cos u| \, du \\
                & = 8 \cdot \int_{-\frac{\pi}{2}}^{\frac{\pi}{2}} \cos u \, du \\
                & = 8 \cdot [\sin u]_{-\frac{\pi}{2}}^{\frac{\pi}{2}} \\
                & = 16
            \end{align*}
            \item \begin{align*}
                \int_0^{8} \sqrt{ \sum_{i=1}^{n} (\gamma^\prime_i(\theta))^2 } & = \int_0^{8} \sqrt{\frac{9}{4}t + 9 - \frac{9}{4} t} \\
                & = \int_0^{8} \sqrt{9} \\
                & = \int_0^{8} 3 \\
                & = [3t]_0^8 \\
                & = 24
            \end{align*}
        \end{enumerate}
            \item 
            \begin{enumerate}
                \item\begin{align*}
                \begin{pmatrix}
                    a & b
                \end{pmatrix}
                \cdot
                \begin{pmatrix}
                    \frac{1}{2} & -\frac{\sqrt{3}}{2} \\
                    \frac{\sqrt{3}}{2} & \frac{1}{2}
                \end{pmatrix}
                \cdot
                \begin{pmatrix}
                    a \\
                    b
                \end{pmatrix} & = 
                \begin{pmatrix}
                    a \cdot \frac{1}{2} + b \cdot \frac{\sqrt{3}}{2} & 
                    -a \cdot \frac{\sqrt{3}}{2} + b \cdot \frac{1}{2}
                \end{pmatrix}
                \cdot
                \begin{pmatrix}
                    a \\
                    b
                \end{pmatrix}\\
                & = 
                a^2 \cdot \frac{1}{2} -ab \cdot \frac{\sqrt{3}}{2} + 
                b^2 \cdot \frac{1}{2} + ab \cdot \frac{\sqrt{3}}{2} \\
                & = 
                a^2 \cdot \frac{1}{2} + 
                b^2 \cdot \frac{1}{2} \\
                & \Rightarrow \text{The Matrix is positivly definite}
            \end{align*}
        \item \begin{align*}
            0 & = det (
                \begin{pmatrix}
                    t - 2 & x \\
                    0 & t - 2
                \end{pmatrix}
            ) \\
            & = (t - 2)^2 \quad | \sqrt{}\\
            & \Leftrightarrow 0 = t - 2 \quad | + 2 \\
            & \Leftrightarrow 2 = t \\
            & \Rightarrow \text{The Matrix is positivly definite for every value of x}
        \end{align*}
        \item \begin{align*}
            0 & = det (
                \begin{pmatrix}
                    t - 0,5 & 0 \\
                    0 & t - x
                \end{pmatrix}
            ) \\
            & = (t - 0,5) \cdot (t - x) \\
            \Rightarrow t_1 & = 0,5 \\
            t_2 & = x \\
            & \Rightarrow \text{For } x \in \R^+/\{0\} \text{ the matrix is positivly define,}\\ &\text{for } x = {0} \text{ it is positivly semi definite and for } x \in \R^-/\{0\} \text{ it is undefinite}
         \end{align*}
        \end{enumerate}
    \item 
    \begin{enumerate}
        \item 
    Explicit\\
    Implicit: $0 = \begin{pmatrix}
        4 \\ 
        10 \\
        -2
    \end{pmatrix}^T
    \cdot
    \begin{pmatrix}
        x \\
        y \\ 
        z 
    \end{pmatrix}
    - 15
    $ \\
    Parametric: $\begin{pmatrix}
        0 \\
        0 \\
        -7,5
    \end{pmatrix} +
    \begin{pmatrix}
        1 \\
        0 \\
        2 \\
    \end{pmatrix} \cdot t +
    \begin{pmatrix}
        0\\
        1\\
        5
    \end{pmatrix} \cdot r
    $
    \item Implicit \\
    Explicit: $z = -(x + y) - \sqrt{75}$ \\
    Parametric: $\begin{pmatrix}
        0 \\
        0 \\
        -\sqrt{75} 
    \end{pmatrix} +
    \begin{pmatrix}
        1 \\
        0 \\
        -1 \\
    \end{pmatrix} \cdot t +
    \begin{pmatrix} 
        0\\
        1\\
        -1
    \end{pmatrix} \cdot r
    $ 
    \item Implicit \\
    Explicit: $z = \sqrt{25 - (x - 6)^2 - (y - 3)^2}$\\
    Parametric: $
    \begin{pmatrix} 
        x\\
        y\\
        z
    \end{pmatrix} =
    \begin{pmatrix}
        (25 \sin \phi \cos \theta) + 6 \\
        (25 \sin \phi \sin \theta) + 3\\
        (25 \cos \phi)
    \end{pmatrix}
    for 0 \leq \phi, \theta \leq 2 \pi
    $
\end{enumerate}
\item \begin{enumerate}
    \item $
        \begin{pmatrix}
            2x_1 & 0 \\
            0 & 2x_2
        \end{pmatrix}
    $
    \item $
        \begin{pmatrix}
            3x_1^2x_2^2 - x_2^3 & x_1^32x_2 - x_13x_2^2 \\
            2x_1 - x_2^3 & - x_13x_2^2
        \end{pmatrix}
    $
    \item$
    \begin{pmatrix}
        4x_1 & -x_3\sin(x_2x_3) & -x_2\sin(x_2x_3)\\
        8x_1 & 1260x_3^2 &  4\\
        -x_2e^{-x_1x_2}& -x_1e^{-x_1x_2} & 20
    \end{pmatrix}
$
\end{enumerate}
\end{enumerate}

\end{document}