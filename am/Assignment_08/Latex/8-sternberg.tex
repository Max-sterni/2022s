\documentclass[]{book}

%These tell TeX which packages to use.
\usepackage{array,epsfig}
\usepackage{amsmath}
\usepackage{amsfonts}
\usepackage{amssymb}
\usepackage{amsxtra}
\usepackage{amsthm}
\usepackage{mathrsfs}
\usepackage{color}

%Here I define some theorem styles and shortcut commands for symbols I use often
\theoremstyle{definition}
\newtheorem{defn}{Definition}
\newtheorem{thm}{Theorem}
\newtheorem{cor}{Corollary}
\newtheorem*{rmk}{Remark}
\newtheorem{lem}{Lemma}
\newtheorem*{joke}{Joke}
\newtheorem{ex}{Example}
\newtheorem*{soln}{Solution}
\newtheorem{prop}{Proposition}

\newcommand{\lra}{\longrightarrow}
\newcommand{\ra}{\rightarrow}
\newcommand{\surj}{\twoheadrightarrow}
\newcommand{\graph}{\mathrm{graph}}
\newcommand{\bb}[1]{\mathbb{#1}}
\newcommand{\Z}{\bb{Z}}
\newcommand{\Q}{\bb{Q}}
\newcommand{\R}{\bb{R}}
\newcommand{\C}{\bb{C}}
\newcommand{\N}{\bb{N}}
\newcommand{\M}{\mathbf{M}}
\newcommand{\m}{\mathbf{m}}
\newcommand{\MM}{\mathscr{M}}
\newcommand{\HH}{\mathscr{H}}
\newcommand{\Om}{\Omega}
\newcommand{\Ho}{\in\HH(\Om)}
\newcommand{\bd}{\partial}
\newcommand{\del}{\partial}
\newcommand{\bardel}{\overline\partial}
\newcommand{\textdf}[1]{\textbf{\textsf{#1}}\index{#1}}
\newcommand{\img}{\mathrm{img}}
\newcommand{\ip}[2]{\left\langle{#1}
{#2}\right\rangle}
\newcommand{\inter}[1]{\mathrm{int}{#1}}
\newcommand{\exter}[1]{\mathrm{ext}{#1}}
\newcommand{\cl}[1]{\mathrm{cl}{#1}}
\newcommand{\ds}{\displaystyle}
\newcommand{\vol}{\mathrm{vol}}
\newcommand{\cnt}{\mathrm{ct}}
\newcommand{\osc}{\mathrm{osc}}
\newcommand{\LL}{\mathbf{L}}
\newcommand{\UU}{\mathbf{U}}
\newcommand{\support}{\mathrm{support}}
\newcommand{\AND}{\;\wedge\;}
\newcommand{\OR}{\;\vee\;}
\newcommand{\Oset}{\varnothing}
\newcommand{\st}{\ni}
\newcommand{\wh}{\widehat}

%Pagination stuff.
\setlength{\topmargin}{-.3 in}
\setlength{\oddsidemargin}{0in}
\setlength{\evensidemargin}{0in}
\setlength{\textheight}{9.in}
\setlength{\textwidth}{6.5in}
\pagestyle{empty}



\begin{document}


\begin{center}
{\Large Angewandte Mathimatik \hspace{0.5cm} Sheet 8}
\textbf{Maximilian von Sternberg} %You should put your name here
\end{center}

\vspace{0.2 cm}

\begin{enumerate}
    \item \begin{enumerate}
        \item \begin{align*}
            f^\prime(x) = & \frac{2e^{2x} \cdot (e^{2x} + 1) - (e^{2x} - 1) \cdot 2e^{2x}}{(e^{2x} + 1)^2} \\
            f^\prime(x) = & \frac{2e^{2x} \cdot (e^{2x} + 1 - e^{2x} + 1)}{(e^{2x} + 1)^2} \\
            f^\prime(x) = & \frac{4e^{2x}}{(e^{2x} + 1)^2} \\
            f^\prime(x) = & \frac{4e^{2x}}{e^{4x} + 2e^{2x} + 1} \\
        \end{align*}
        \item Julia
        \item Julia
        \item Julia
        \item It is more likely, that the mean value of a secant is between the values, used in the calculation. Espacially when the functions second differential is a small value, meaning that the rate of change is less significant. 
    \end{enumerate}
    \item \begin{enumerate}
        \item \begin{align*}
            [\log(\cosh(x)) + C]^5_{-1} = & \log(\cosh(5)) - \log(\cosh(-1))\\
            = & 3.87
        \end{align*}
        \item \begin{align*}
            f(x) = & \frac{\sinh(x)}{\cosh(x)} \\
             = & \tanh(x)
        \end{align*}
        \item Julia
        \item Julia
        \item Julia
        \item The Simpson Method performes best, because it was optimized to waste as little space as possible.
    \end{enumerate}
    \item \begin{enumerate}
        \item \begin{align*}
            \frac{\partial}{\partial t}u(x_i, t_n) = & \frac{u(x_i, t_{n + 1}) - u(x_i, t_n)}{\tau} \\
            \frac{\partial^2}{\partial x^2}u(x_i, t_n) = & \frac{u(x_i, t_{n + 1}) - u(x_i, t_n)}{\tau} \\
            \frac{u(x_{i + 1}, t_n) - 2u(x_{i}, t_n) + u(x_{i - 1}, t_n)}{h^2} = & \frac{u(x_i, t_{n + 1}) - u(x_i, t_n)}{\tau} \quad | \cdot \tau\\
            \Leftrightarrow \tau \frac{u(x_{i + 1}, t_n) - 2u(x_{i}, t_n) + u(x_{i - 1}, t_n)}{h^2} = & u(x_i, t_{n + 1}) - u(x_i, t_n) \quad | + u(x_i, t_n)\\
            \Leftrightarrow \tau \frac{u(x_{i + 1}, t_n) - 2u(x_{i}, t_n) + u(x_{i - 1}, t_n)}{h^2} + u(x_i, t_n) = & u(x_i, t_{n + 1})\\
        \end{align*}
    \end{enumerate}
    \item \begin{enumerate}
        \item \begin{align*}
            2x^6 - 5x^5 + x^2 - 6x + 1 = & (2x) \cdot x^5 - 5x^5 + x^2 - 6x + 1 \\
            = & ((2x) - 5) \cdot x^5 + x^2 - 6x + 1 \\
            = & (((2x) - 5) \cdot x^3 + 1) \cdot x^2 - 6x + 1 \\
            = & (((2x) - 5) \cdot x^3 + 1) \cdot x -6) \cdot x + 1 \\
        \end{align*}
        \item \begin{align*}
            &12x^7 + 2x^4y^6 + x^3y^4 - 3x^2 - 7x^2y^4 - 2 + 9y + 2y^5 \\= & (12x^7 - 3x^2) + (2y^5 + 9y) + (2x^4y^6 + x^3y^4 - 7x^2y^4) - 2 \\
            = & ((12x^5 - 3) \cdot x^2) + ((2y^4 + 9) \cdot y) + (((2xy^2 + 1) \cdot x - 7) \cdot x^2y^4) - 2 \\
        \end{align*}
        \item Julia
    \end{enumerate}
\end{enumerate}


\end{document}