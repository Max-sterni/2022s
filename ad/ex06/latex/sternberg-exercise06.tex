\documentclass[]{book}

%These tell TeX which packages to use.
\usepackage{array,epsfig}
\usepackage{amsmath}
\usepackage{amsfonts}
\usepackage{amssymb}
\usepackage{amsxtra}
\usepackage{amsthm}
\usepackage{mathrsfs}
\usepackage{color}

%Here I define some theorem styles and shortcut commands for symbols I use often
\theoremstyle{definition}
\newtheorem{defn}{Definition}
\newtheorem{thm}{Theorem}
\newtheorem{cor}{Corollary}
\newtheorem*{rmk}{Remark}
\newtheorem{lem}{Lemma}
\newtheorem*{joke}{Joke}
\newtheorem{ex}{Example}
\newtheorem*{soln}{Solution}
\newtheorem{prop}{Proposition}

\newcommand{\lra}{\longrightarrow}
\newcommand{\ra}{\rightarrow}
\newcommand{\surj}{\twoheadrightarrow}
\newcommand{\graph}{\mathrm{graph}}
\newcommand{\bb}[1]{\mathbb{#1}}
\newcommand{\Z}{\bb{Z}}
\newcommand{\Q}{\bb{Q}}
\newcommand{\R}{\bb{R}}
\newcommand{\C}{\bb{C}}
\newcommand{\N}{\bb{N}}
\newcommand{\M}{\mathbf{M}}
\newcommand{\m}{\mathbf{m}}
\newcommand{\MM}{\mathscr{M}}
\newcommand{\HH}{\mathscr{H}}
\newcommand{\Om}{\Omega}
\newcommand{\Ho}{\in\HH(\Om)}
\newcommand{\bd}{\partial}
\newcommand{\del}{\partial}
\newcommand{\bardel}{\overline\partial}
\newcommand{\textdf}[1]{\textbf{\textsf{#1}}\index{#1}}
\newcommand{\img}{\mathrm{img}}
\newcommand{\ip}[2]{\left\langle{#1}
{#2}\right\rangle}
\newcommand{\inter}[1]{\mathrm{int}{#1}}
\newcommand{\exter}[1]{\mathrm{ext}{#1}}
\newcommand{\cl}[1]{\mathrm{cl}{#1}}
\newcommand{\ds}{\displaystyle}
\newcommand{\vol}{\mathrm{vol}}
\newcommand{\cnt}{\mathrm{ct}}
\newcommand{\osc}{\mathrm{osc}}
\newcommand{\LL}{\mathbf{L}}
\newcommand{\UU}{\mathbf{U}}
\newcommand{\support}{\mathrm{support}}
\newcommand{\AND}{\;\wedge\;}
\newcommand{\OR}{\;\vee\;}
\newcommand{\Oset}{\varnothing}
\newcommand{\st}{\ni}
\newcommand{\wh}{\widehat}

%Pagination stuff.
\setlength{\topmargin}{-.3 in}
\setlength{\oddsidemargin}{0in}
\setlength{\evensidemargin}{0in}
\setlength{\textheight}{9.in}
\setlength{\textwidth}{6.5in}
\pagestyle{empty}



\begin{document}


\begin{center}
{\Large Algorithmen und Datenstruckturen \hspace{0.5cm} Sheet 6}
\textbf{Maximilian von Sternberg} %You should put your name here
\end{center}
\begin{enumerate}
    \item 1
    \item sternberg-exercise06-02-1/2.pdf
    \item \begin{enumerate}
        \item sternberg-exercise06-03.pdf
        \item 
    \end{enumerate}
    \item \begin{enumerate}
        \item \begin{enumerate}
            \item Eine priority queue, da neue Anfragen direkt mit der insert Methode eingesetzt werden können und zu bearbeitende Anfragen mit removeMin entnommen werden können.
            \item Eine Positionsbasierte Liste, da dadurch die Server einen Nachbar haben können. Wenn der Server beschäftigt ist, kann mit next einfach der nächste Server genommen werden. Außerdem kann die Serverliste zyklisch implementiert werden, wenn der letzte Server den ersten Server als Nachbar hat.
        \end{enumerate}
        \item \begin{enumerate}
            \item Eine unsortierte Liste, da man beim hinzufügen der Anfragen eine konstante Komplexität hat und die Berechnungen, welches das minimale Element ist, nur am Ende geschehen.
            \item Eine sortierte Liste, da man herrausnehmen der Elemente, eine konstante Komplexität hat, da die Berechnungen bereits beim Einfügen getätigt wereden. 
        \end{enumerate}
    \end{enumerate}
\end{enumerate}
\vspace{0.2 cm}


\end{document}